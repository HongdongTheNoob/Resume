% !TEX program = xelatex
%%%%%%%%%%%%%%%%%%%%%%%%%%%%%%%%%%%%%%%%%
% Medium Length Professional CV
% LaTeX Template
% Version 2.0 (8/5/13)
%
% This template has been downloaded from:
% http://www.LaTeXTemplates.com
%
% Original author:
% Trey Hunner (http://www.treyhunner.com/)
%
% Important note:
% This template requires the resume.cls file to be in the same directory as the
% .tex file. The resume.cls file provides the resume style used for structuring the
% document.
%
%%%%%%%%%%%%%%%%%%%%%%%%%%%%%%%%%%%%%%%%%

%----------------------------------------------------------------------------------------
%	PACKAGES AND OTHER DOCUMENT CONFIGURATIONS
%----------------------------------------------------------------------------------------

\documentclass{resume} % Use the custom resume.cls style

\usepackage[left=0.75in,top=0.6in,right=0.75in,bottom=0.6in]{geometry} % Document margins

\usepackage{fontspec, xunicode, xltxtra}
\usepackage{xeCJK}

\setmainfont{Segoe UI}
\setCJKmainfont{HarmonyOS Sans SC}

\usepackage[ocgcolorlinks]{hyperref}
\hypersetup{
	colorlinks   = true, %Colours links instead of ugly boxes
	urlcolor     = blue, %Colour for external hyperlinks
	linkcolor    = blue, %Colour of internal links
	citecolor   = red %Colour of citations
}

\newif\ifen
\newif\ifzh

\newcommand{\en}[1]{\ifen#1\fi}
\newcommand{\zh}[1]{\ifzh#1\fi}

\entrue

\name{\en{QIN Hongdong}\zh{覃泓胨}} % Your name
\address{(+86)13603087140 \\ hongdongdonald@gmail.com} % Your phone number and email

\begin{document}

%----------------------------------------------------------------------------------------
%	PROFILE SECTION
%----------------------------------------------------------------------------------------

\begin{rSection}{\en{Profile}\zh{简介}}
\en{
	PhD in electrical and electronic engineering and senior engineer on technical standardisation. Focusing on video coding development, patents investigations and standardisation practices.
}
\zh{
	电机电子工程博士,技术标准工程师。主要工作方向为视频编解码技术研发,专利申请,标准推动等。
}
\end{rSection}

%----------------------------------------------------------------------------------------
%	EDUCATION SECTION
%----------------------------------------------------------------------------------------

\begin{rSection}{\en{Education}\zh{教育背景}}
	{\bf\en{The University of Hong Kong}\zh{香港大学}} \hfill {2013-09 -- 2020-06}\\
	{\en{Doctor of Philosophy}\zh{哲学博士}} \hfill {\en{Hong Kong}\zh{香港}}\\
	\en{Thesis: }\zh{毕业论文:}\href{http://hdl.handle.net/10722/318421}{\textit{Novel Techniques for Depth Map Compression}}

	{\bf\en{National University of Singapore}\zh{新加坡国立大学}} \hfill {2008-08 -- 2008-12}\\
	{\en{Exchange Programme, Electrical and Computer Engineering}\zh{电机与电脑工程系交换生}} \hfill {\en{Singapore}\zh{新加坡}}

	{\bf\en{The Hong Kong Polytechnic University}\zh{香港理工大学}} \hfill {2006-09 -- 2011-08}\\
	{\en{Bachelor of Engineering in Electronic and Information Engineering}\zh{工学学士,电子及资讯工程学}} \hfill {\en{Hong Kong}\zh{香港}}\\
	\en{Graduated with 1st Class Honours}\zh{甲等荣誉}\\
	\en{Included 12-month industrial training}\zh{包含12个月实习训练}

\end{rSection}

% ----------------------------------------------------------------------------------------
% 	WORK EXPERIENCE SECTION
% ----------------------------------------------------------------------------------------

\begin{rSection}{\en{Work Experience}\zh{工作经历}}
	\begin{rSubsection}{TCL}
		{2023-01 -- \en{Now}\zh{现在}}
		{\en{Senior Engineer, Technical Standardisation at Eagle Lab}\zh{鸿鹄实验室,技术标准高级工程师}}
		{\en{Shenzhen}\zh{深圳}}
		\en{\item Perform technical pre-research on audio/video coding technologies and analyse patents}
		\zh{\item 视频编解码技术预研,相关专利分析}
		\en{\item Develop new video compression methods for standardisation purposes}
		\zh{\item 开发视频编解码新方法,参与标准制定}
	\end{rSubsection}
	
	\begin{rSubsection}{\en{Shenzhen University}\zh{深圳大学}}
		{2020-10 -- 2022-12}
		{\en{Post-doctoral Researcher at School of Biomedical Engineering}\zh{生物医学工程学院,博士后研究员}}
		{\en{Shenzhen}\zh{深圳}}
		\en{\item Worked on electroencephalography (EEG) related cognitive neuroscience investigations}        
		\zh{\item 基于视频脑电的认识神经科学研究}
		\en{\item Worked on project: localisation of intracranial EEG electrode contacts in CT scans}
        \zh{\item 负责研发CT扫描中的颅内脑电电极定位算法}
	\end{rSubsection}
	
	\begin{rSubsection}{\en{The University of Hong Kong}\zh{香港大学}}
		{2016-07 -- 2019-06}
		{\en{Research Assistant at Department of Electrical and Electronic Engineering}\zh{电机电子工程系,研究助理}}
		{\en{Hong Kong}\zh{香港}}
		\en{\item Researched on 3D video coding}
        \zh{\item 研发三维视频编解码方法}
		\en{\item Depth map codec outperformed 3D-HEVC}
        \zh{\item 自行开发的深度图压缩方法的效能超越同期的3D-HEVC}
	\end{rSubsection}
	
	\begin{rSubsection}{\en{Marvel Digital Limited}\zh{万维数码}}
		{2015-08 -- 2017-12}
		{\en{Consultant}\zh{工程咨询师}}
		{\en{Hong Kong}\zh{香港}}
		\en{\item Developed 3D video coding methods and deal with intellectual property paperworks} 
        \zh{\item 参与三维视频编解码标准竞投,撰写专利文书}
	\end{rSubsection}
	
	\begin{rSubsection}{\en{The University of Hong Kong}\zh{香港大学}}
		{2011-08 -- 2015-08}
		{\en{Research Assistant at Department of Electrical and Electronic Engineering}\zh{电机电子工程系,研究助理}}
		{\en{Hong Kong}\zh{香港}}
		\en{\item Worked in collaboration with Audio Video coding Standard workgroup (AVS)  of China}
        \zh{\item 参与中国数字音视频编解码技术标准项目(AVS)的研发}
		\en{\item Focused on video codec design and acceleration, 3D video content generation}
        \zh{\item 研发视频编解码加速,三维视频内容生成方法}
		\en{\item Achieved real-time AVS video encoding and decoding in ARM+DSP low power platform}
        \zh{\item AVS-1在ARM+DSP平台上可以达到实时的编解码}
		\en{\item Developed a semi-automatic image segmentation tool}
        \zh{\item 开发半自动的图像分割和抠图工具}
	\end{rSubsection}
	
	\begin{rSubsection}{\en{Solomon Systech Limited}\zh{晶门科技}}
		{2009-06 -- 2010-05}
		{\en{Engineering Trainee at Design Engineering Department}\zh{设计工程部,工程实习生}}
		{\en{Hong Kong}\zh{香港}}
		\en{\item Worked on image compression and coding algorithms for mobile device display system}
        \zh{\item 移动端显示设备上的快速图像编解码}
		\en{\item Designed specifications and develop supporting softwares for new microcontroller product}
        \zh{\item 参与开发和测试新微控制器产品的汇编环境等工具,测试指令集}
	\end{rSubsection}
\end{rSection}

%----------------------------------------------------------------------------------------
%	PROJECTS SECTION
%----------------------------------------------------------------------------------------

\begin{rSection}{\en{Projects}\zh{参与项目}}
    \begin{rSubsection}{音视频编解码技术研究}{2023年01月 -- 现在}{}{}
        \item 视频编解码技术预研,相关专利分析
		\item 开发视频编解码新方法,专利申请,参与标准制定
    \end{rSubsection}
	
	\begin{rSubsection}{\en{Intracranial Video-electroencephalography}\zh{颅内视频脑电智能计算}}
		{2022-04 -- 2022-10}{}{}
		\en{\item Localised EEG electrode contacts from CT scans for pre-surgical planning}
        \zh{\item 主要负责的工作是从病人头颅CT扫描中寻找植入的脑电电极}
		\en{\item Primarily achieved localisation and automatic grouping}
        \zh{\item 初步实现电极的检测、定位、自动分组}
	\end{rSubsection}

	\begin{rSubsection}{\en{Auxiliary Video Data Compression and View Synthesis System}\zh{辅助视频数据压缩和视角生成系统}}
		{2015-05 -- 2018-12}{}{}
		\en{\item Designed compression algorithms for depth maps}
        \zh{\item 负责三维视频中的深度图压缩方法的设计,并开发了一套原创的深度图像和视频的压缩方法}
		\en{\item Achieved a compression performance on par with 3D-HEVC or better without texture images}
        \zh{\item 在不依赖纹理图像的情况下压缩深度图像,其压缩效能与3D-HEVC相当或更佳}
		\en{\item Developed a GPU-accelerated software for 1080p real-time decoding}
        \zh{\item 为压缩方法在GPU上设计了一套算法流程,达到1080p视频实时解码}
	\end{rSubsection}
		
	\begin{rSubsection}{\en{3D Video Content Generation and Processing System}\zh{三维视频内容生成及处理系统}}
		{2013-05 -- 2015-08}{}{}
		\en{\item Developed a semi-automatic image segmentation and data management tool}
        \zh{\item 开发半自动的图像分割和抠图算法}
		\en{\item Designed related image segmentation algorithms}
        \zh{\item 为制作三维视频的合作方提供了一个给图像进行物体切割,分层,组织数据的操作工具}
	\end{rSubsection}
		
	\begin{rSubsection}{\en{Real-time AVS-1 Video Codec on Mobile Platform}\zh{AVS-1视频编码标准在移动平台的实现}}
		{2011-08 -- 2013-05}{}{}
		\en{\item Ported AVS video codec onto mobile-class ARM+DSP platform}
        \zh{\item 将AVS的视频编解码移植到与移动设备功耗相当的ARM+DSP平台}
		\en{\item Achieved real-time 720p encoding and decoding}
        \zh{\item 达到720p视频的实时编码和解码}
	\end{rSubsection}

\end{rSection}

%----------------------------------------------------------------------------------------
%	PUBLICATION AND PATENT SECTION
%----------------------------------------------------------------------------------------

\iffalse
\begin{rSection}{\en{Publications and Patents}\zh{发表的论文与专利}}

Z. Lin, \textbf{H. Qin} and S. C. Chan, ``A New Probabilistic Representation of Color Image Pixels and Its Applications,'' in \textit{IEEE Transactions on Image Processing}, vol. 28, no. 4, pp. 2037-2050, April 2019.

``A depth discontinuity-based method for efficient intra coding for depth videos'', WO 2017/020808, February 09, 2017.

``Systems and Methods for multiple layer representation of depth map for intra coding'', Hong Kong Short-term Patent Application No. 19124682.6

\end{rSection}
\fi

%----------------------------------------------------------------------------------------
%	TECHNICAL STRENGTHS SECTION
%----------------------------------------------------------------------------------------

\begin{rSection}{\en{Other Skills}\zh{其它技能}}

\begin{tabular}{ @{} >{\bfseries}l @{\hspace{2ex}} l }
\en{Natural Languages}\zh{人类语言} & 
\en{Mandarin (native), Cantonese (native), English (fluent), French (elementary)} 
\zh{普通话(母语),粤语(母语),英语(熟练流利),法语(爱好,会一点)}\\
\en{Computer Languages}\zh{计算机语言} & C/C++, Python, MatLab \\
\en{Tools}\zh{开发工具} & Visual Studio, VSCode, Eclipse, Anaconda platform
\end{tabular}

\end{rSection}

%----------------------------------------------------------------------------------------

\end{document}
