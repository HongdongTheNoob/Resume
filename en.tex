% !TEX program = xelatex
%%%%%%%%%%%%%%%%%%%%%%%%%%%%%%%%%%%%%%%%%
% Medium Length Professional CV
% LaTeX Template
% Version 2.0 (8/5/13)
%
% This template has been downloaded from:
% http://www.LaTeXTemplates.com
%
% Original author:
% Trey Hunner (http://www.treyhunner.com/)
%
% Important note:
% This template requires the resume.cls file to be in the same directory as the
% .tex file. The resume.cls file provides the resume style used for structuring the
% document.
%
%%%%%%%%%%%%%%%%%%%%%%%%%%%%%%%%%%%%%%%%%

%----------------------------------------------------------------------------------------
%	PACKAGES AND OTHER DOCUMENT CONFIGURATIONS
%----------------------------------------------------------------------------------------

\documentclass{resume} % Use the custom resume.cls style

\usepackage[left=0.75in,top=0.6in,right=0.75in,bottom=0.6in]{geometry} % Document margins

\usepackage{fontspec, xunicode, xltxtra}
\usepackage{xeCJK}

\setmainfont{Calibri}
\setCJKmainfont{HarmonyOS Sans SC}

\usepackage[ocgcolorlinks]{hyperref}
\hypersetup{
	colorlinks   = true, %Colours links instead of ugly boxes
	urlcolor     = blue, %Colour for external hyperlinks
	linkcolor    = blue, %Colour of internal links
	citecolor   = red %Colour of citations
}

\name{QIN Hongdong} % Your name
\address{(+86)13603087140 \\ hongdongdonald@gmail.com} % Your phone number and email

\begin{document}

%----------------------------------------------------------------------------------------
%	PROFILE SECTION
%----------------------------------------------------------------------------------------

\begin{rSection}{Profile}

PhD in electrical and electronic engineering and senior engineer on technical standardisation. Focusing on video coding development, patents investigations and standardisation practices.

\end{rSection}

%----------------------------------------------------------------------------------------
%	EDUCATION SECTION
%----------------------------------------------------------------------------------------

\begin{rSection}{Education}

{\bf The University of Hong Kong} \hfill {Sep 2013 -- Jun 2020} \\ 
Doctor of Philosophy \\
Thesis title: \href{http://hdl.handle.net/10722/318421}{\textit{Novel Techniques for Depth Map Compression}}

{\bf National University of Singapore} \hfill {Aug 2008 -- Dec 2008} \\ 
Exchange Student, Electrical and Computer Engineering

{\bf The Hong Kong Polytechnic University} \hfill {Sep 2006 -- Aug 2011} \\ 
Bachelor of Engineering in Electronic and Information Engineering, 1st Class Honours, includes 12-month industrial training

\end{rSection}

%----------------------------------------------------------------------------------------
%	WORK EXPERIENCE SECTION
%----------------------------------------------------------------------------------------

\begin{rSection}{Work Experience}
	\begin{rSubsection}{TCL Industries}{Jan 2023 -- Now}{Senior Engineer, Technical Standardisation at Eagle Lab}{Shenzhen}
		\item Perform technical pre-research on audio/video coding technologies and analyse relevant patents
		\item Develop new video compression techniques for patent applications and standardisation
	\end{rSubsection}
	
	\begin{rSubsection}{Shenzhen University}{Oct 2020 -- Dec 2022}{Post-doctoral Researcher at School of Biomedical Engineering}{Shenzhen}
		\item Worked on electroencephalography (EEG) related cognitive neuroscience investigations
		\item Worked on project: localisation of intracranial EEG electrode contacts in CT scans
	\end{rSubsection}
	
	\begin{rSubsection}{The University of Hong Kong}{Jul 2016 -- Jun 2019}{Research Assistant at Department of Electrical and Electronic Engineering}{Hong Kong}
		\item Worked on 3D video coding
		\item Our depth map codec outperformed 3D-HEVC at that era
	\end{rSubsection}
	
	\begin{rSubsection}{Marvel Digital Limited}{Aug 2015 -- Dec 2017}{Consultant}{Hong Kong}
		\item Developed 3D video coding methods and deal with intellectual property paperworks 
	\end{rSubsection}
	
	\begin{rSubsection}{The University of Hong Kong}{Aug 2011 -- Aug 2015}{Research Assistant at Department of Electrical and Electronic Engineering}{Hong Kong}
		\item Worked in collaboration with Audio Video coding Standard workgroup (AVS)  of China
		\item Focused on video codec design and acceleration, 3D video content generation
		\item Achieved real-time AVS video encoding and decoding in ARM+DSP low power platform
		\item Developed a semi-automatic image segmentation tool
	\end{rSubsection}
	
	\begin{rSubsection}{Solomon Systech Limited}{Jun 2009 -- May 2010}{Engineering Trainee at Design Engineering Department}{Hong Kong}
		\item Worked on image compression and coding algorithms for mobile device display system
		\item Designed specifications and develop supporting softwares for new microcontroller product
	\end{rSubsection}
\end{rSection}

%----------------------------------------------------------------------------------------
%	PROJECTS SECTION
%----------------------------------------------------------------------------------------

\begin{rSection}{Projects}

	\begin{rSubsection}{Intracranial Video-electroencephalography}{Apr 2022 -- Oct 2022}{}{}
		\item Localise EEG electrode contacts from CT scans for pre-surgical planning
		\item Primarily achieved localisation and automatic grouping
	\end{rSubsection}

	\begin{rSubsection}{Auxiliary Video Data Compression and View Synthesis System}{May 2015 -- Nov 2018}{}{}
		\item Designed compression algorithms for depth images and videos
		\item Presented a compression performance on par with 3D-HEVC or better without referring to texture images
		\item Developed a GPU-accelerated software for 1080p real-time decoding
	\end{rSubsection}
		
	\begin{rSubsection}{3D Video Content Generation and Processing System}{May 2013 -- Aug 2015}{}{}
		\item Developed image segmentation algorithms
		\item Developed a semi-automatic image segmentation and data management tool for our collaboration partner
	\end{rSubsection}
		
	\begin{rSubsection}{Real-time AVS-1 Video Codec on ARM+DSP Platform}{Aug 2011 -- May 2013}{}{}
		\item Ported AVS video codec onto mobile-class ARM+DSP platform
		\item Reached real-time 720p encoding and decoding with computation workload management and asynchronous processing
	\end{rSubsection}

\end{rSection}

%----------------------------------------------------------------------------------------
%	PUBLICATION AND PATENT SECTION
%----------------------------------------------------------------------------------------

\iffalse
\begin{rSection}{Publications and Patents}

Z. Lin, \textbf{H. Qin} and S. C. Chan, ``A New Probabilistic Representation of Color Image Pixels and Its Applications,'' in \textit{IEEE Transactions on Image Processing}, vol. 28, no. 4, pp. 2037-2050, April 2019.

``A depth discontinuity-based method for efficient intra coding for depth videos'', WO 2017/020808, February 09, 2017.

``Systems and Methods for multiple layer representation of depth map for intra coding'', Hong Kong Short-term Patent Application No. 19124682.6

\end{rSection}
\fi

%----------------------------------------------------------------------------------------
%	TECHNICAL STRENGTHS SECTION
%----------------------------------------------------------------------------------------

\begin{rSection}{Technical Strengths}

\begin{tabular}{ @{} >{\bfseries}l @{\hspace{2ex}} l }
Natural Languages & Mandarin (native), Cantonese (native), English (fluent) \\
Computer Languages & C/C++, MatLab, Python \\
Tools & Visual Studio, VSCode, Eclipse, Anaconda platform
\end{tabular}

\end{rSection}

%----------------------------------------------------------------------------------------

\end{document}
