% !TEX program = xelatex
%%%%%%%%%%%%%%%%%%%%%%%%%%%%%%%%%%%%%%%%%
% Medium Length Professional CV
% LaTeX Template
% Version 2.0 (8/5/13)
%
% This template has been downloaded from:
% http://www.LaTeXTemplates.com
%
% Original author:
% Trey Hunner (http://www.treyhunner.com/)
%
% Important note:
% This template requires the resume.cls file to be in the same directory as the
% .tex file. The resume.cls file provides the resume style used for structuring the
% document.
%
%%%%%%%%%%%%%%%%%%%%%%%%%%%%%%%%%%%%%%%%%

%----------------------------------------------------------------------------------------
%	PACKAGES AND OTHER DOCUMENT CONFIGURATIONS
%----------------------------------------------------------------------------------------

\documentclass{resume} % Use the custom resume.cls style

\usepackage[left=0.75in,top=0.6in,right=0.75in,bottom=0.6in]{geometry} % Document margins

\usepackage{fontspec, xunicode, xltxtra}
\usepackage{xeCJK}

\setmainfont{Segoe UI}
\setCJKmainfont{HarmonyOS Sans SC}

\usepackage[ocgcolorlinks]{hyperref}
\hypersetup{
	colorlinks   = true, %Colours links instead of ugly boxes
	urlcolor     = blue, %Colour for external hyperlinks
	linkcolor    = blue, %Colour of internal links
	citecolor   = red %Colour of citations
}

\name{覃泓胨} % Your name
\address{(+86)~13603087140 \\ hongdongdonald@gmail.com} % Your phone number and email


\begin{document}

%----------------------------------------------------------------------------------------
%	PROFILE SECTION
%----------------------------------------------------------------------------------------

\begin{rSection}{简介}

电机电子工程博士,技术标准工程师。主要工作方向为视频编解码技术研发,专利申请,标准推动等。

\end{rSection}

%----------------------------------------------------------------------------------------
%	EDUCATION SECTION
%----------------------------------------------------------------------------------------

\begin{rSection}{教育经历}

{\bf 香港大学} \hfill {2013年09月 -- 2020年06月} \\ 
哲学博士,电机电子工程 \\
毕业论文: \href{http://hdl.handle.net/10722/318421}{\textit{Novel Techniques for Depth Map Compression}}

{\bf 新加坡国立大学} \hfill {2008年08月 -- 2008年12月} \\ 
电机与电脑工程系交换生

{\bf 香港理工大学} \hfill {2006年09月 -- 2011年08月} \\ 
工学学士,电子及资讯工程学,甲等荣誉,包含12个月实习训练

\end{rSection}

%----------------------------------------------------------------------------------------
%	WORK EXPERIENCE SECTION
%----------------------------------------------------------------------------------------

\begin{rSection}{工作经历}
    \begin{rSubsection}{TCL}{2023年01月 -- 现在}{鸿鹄实验室,技术标准高级工程师}{深圳}
        \item 视频编解码技术预研,相关专利分析
		\item 开发视频编解码新方法,专利申请,参与标准制定
    \end{rSubsection}

    \begin{rSubsection}{深圳大学}{2020年10月 -- 2022年12月}{医学部生物医学工程学院,博士后研究员}{深圳}
        \item 基于视频脑电的认识神经科学研究
        \item 负责研发CT扫描中的颅内脑电电极定位算法
    \end{rSubsection}

	\begin{rSubsection}{香港大学}{2016年07月 -- 2019年06月}{电机电子工程系,研究助理}{香港}
	\item 研发三维视频编解码方法
	\item 自行开发的深度图压缩方法的效能超越同期的3D-HEVC
	\end{rSubsection}

    \begin{rSubsection}{万维数码}{2015年08月 -- 2017年12月}{工程咨询师}{香港}
        \item 参与三维视频编解码标准竞投,撰写专利文书
    \end{rSubsection}

    \begin{rSubsection}{香港大学}{2011年08月 -- 2015年08月}{电机电子工程系,研究助理}{香港}
        \item 参与中国数字音视频编解码技术标准项目(AVS)的研发
        \item 研发视频编解码加速、在三维视频内容生成方法
        \item AVS-1在ARM+DSP平台上可以达到实时的编解码
        \item 开发半自动的图像分割和抠图工具
    \end{rSubsection}

    \begin{rSubsection}{晶门科技}{2009年06月 -- 2010年05月}{设计工程部,全职工程实习生}{香港}
        \item 移动端显示设备上的快速图像编解码
        \item 参与开发和测试新微控制器产品的汇编环境等工具,测试指令集
    \end{rSubsection}
\end{rSection}

%----------------------------------------------------------------------------------------
%	PROJECTS SECTION
%----------------------------------------------------------------------------------------

\begin{rSection}{参与项目}
    \begin{rSubsection}{音视频编解码技术研究}{2023年01月 -- 现在}{}{}
        \item 视频编解码技术预研,相关专利分析
		\item 开发视频编解码新方法,专利申请,参与标准制定
    \end{rSubsection}

    \begin{rSubsection}{颅内视频脑电智能计算}{2022年04月 -- 2022年10月}{}{}
        \item 主要负责的工作是从病人头颅CT扫描中寻找植入的脑电电极
        \item 初步实现电极的检测、定位、自动分组
    \end{rSubsection}

    \begin{rSubsection}{辅助视频数据压缩和视角生成系统}{2015年05月 -- 2018年11月}{}{}
        \item 负责三维视频中的深度图压缩方法的设计,并开发了一套原创的深度图像和视频的压缩方法
        \item 在不依赖纹理图像的情况下压缩深度图像,其压缩效能与3D-HEVC相当或更佳
        \item 为压缩方法在GPU上设计了一套算法流程,达到1080p视频实时解码
    \end{rSubsection}

    \begin{rSubsection}{三维视频内容生成及处理系统}{2013年05月 -- 2015年05月}{}{}
        \item 开发半自动的图像分割和抠图算法
        \item 为制作三维视频的合作方提供了一个给图像进行物体切割,分层,组织数据的操作工具
    \end{rSubsection}

    \begin{rSubsection}{AVS-1视频编码标准在ARM+DSP平台的实现}{2011年08月 -- 2013年05月}{}{}
        \item 将AVS的视频编解码移植到与移动设备功耗相当的ARM+DSP平台
        \item 通过计算负载重新分配和异步处理,达到720p视频的实时编码和解码
    \end{rSubsection}

\end{rSection}

%----------------------------------------------------------------------------------------
%	PUBLICATION AND PATENT SECTION
%----------------------------------------------------------------------------------------

\iffalse
\begin{rSection}{发表的论文与专利}

Z. Lin, \textbf{H. Qin} and S. C. Chan, ``A New Probabilistic Representation of Color Image Pixels and Its Applications,'' in \textit{IEEE Transactions on Image Processing}, vol. 28, no. 4, pp. 2037-2050, April 2019.

``A depth discontinuity-based method for efficient intra coding for depth videos'', WO 2017/020808, February 09, 2017.

``Systems and methods for multiple layer representation of depth map for intra coding'', Hong Kong Short-term Patent Application No. 19124682.6

\end{rSection}
\fi

%----------------------------------------------------------------------------------------
%	TECHNICAL STRENGTHS SECTION
%----------------------------------------------------------------------------------------

\begin{rSection}{其它技能}

\begin{tabular}{ @{} >{\bfseries}l @{\hspace{6ex}} l }
人类语言 & 普通话(母语),粤语(母语),英语(熟练流利)\\
计算机语言 & C/C++, MatLab, Python \\
开发工具 & Visual Studio, VSCode, Eclipse, Anaconda platform 
\end{tabular}

\end{rSection}

%----------------------------------------------------------------------------------------

\end{document}
