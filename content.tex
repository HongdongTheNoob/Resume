%----------------------------------------------------------------------------------------
%	PROFILE SECTION
%----------------------------------------------------------------------------------------

\begin{rSection}{\en{Profile}\zh{简介}\fr{Profil}}
	\zh{电机电子工程博士,技术标准工程师。主要工作方向为视频编解码技术研发,专利申请,标准推动等。}
	\en{PhD in electrical and electronic engineering and senior engineer on technical standardisation. Focusing on video coding development, patents investigations and standardisation practices.}
	\fr{Docteur en génie électrique et électronique et ingénieur principal en normalisation technique. Axé sur le développement du codage vidéo, les enquêtes sur les brevets et les pratiques de normalisation.}
\end{rSection}

%----------------------------------------------------------------------------------------
%	EDUCATION SECTION
%----------------------------------------------------------------------------------------

\begin{rSection}{\zh{教育背景}\en{Education}\fr{Formation}}
	{\bf\zh{香港大学}\en{The University of Hong Kong}\fr{Université de Hong Kong}} \hfill 
	{2013-09 -- 2020-06}\\
	{\zh{哲学博士}\en{Doctor of Philosophy}\fr{Doctorat}} \hfill 
	{\zh{香港}\lat{Hong Kong}}\\
	\zh{毕业论文:}\en{Thesis: }\fr{Titre de la thèse : }\href{http://hdl.handle.net/10722/318421}{\textit{Novel Techniques for Depth Map Compression}}

	{\bf\zh{新加坡国立大学}\en{National University of Singapore}\fr{Université nationale de Singapour}} \hfill {2008-08 -- 2008-12}\\
	{\zh{电机与电脑工程系交换生}\en{Exchange Programme, Electrical and Computer Engineering}\fr{Programme d'échanges, ingénierie électrique et informatique}} \hfill 
	{\zh{新加坡}\en{Singapore}\fr{Singapour}}

	{\bf\zh{香港理工大学}\en{The Hong Kong Polytechnic University}\fr{Université polytechnique de Hong Kong}} \hfill 
	{2006-09 -- 2011-08}\\
	{\zh{工学学士,电子及资讯工程学}\en{Bachelor of Engineering in Electronic and Information Engineering}\fr{Baccalauréat en ingénierie, ingénierie électronique et informatique}} \hfill
	{\zh{香港}\lat{Hong Kong}}\\
	\zh{甲等荣誉}\en{Graduated with 1st Class Honours}\fr{Avec l'honneur de la 1ère  classe}\\
	\zh{包含12个月实习训练}\en{Included 12-month industrial training}\fr{Avec le stage industriel de 12 mois}
\end{rSection}

% ----------------------------------------------------------------------------------------
% 	WORK EXPERIENCE SECTION
% ----------------------------------------------------------------------------------------

\begin{rSection}{\zh{工作经历}\en{Work Experience}\fr{Expérience professionnelle}}
	\begin{rSubsection}{TCL}
		{2023-01 -- \zh{现在}\en{Now}\fr{Aujourd'hui}}
		{\zh{鸿鹄实验室,技术标准高级工程师}\en{Senior Engineer, Technical Standardisation at Eagle Lab}\fr{Ingénieur sénior, normalisation technique à Eagle Lab}}
		{\zh{深圳}\lat{Shenzhen}}
		\zh{\item 视频编解码技术预研,相关专利分析}
		\en{\item Perform technical pre-research on video coding and analyse relevant patents}
		\fr{\item Pré-recherches techniques sur le codage vidéo et analyser les brevets pertinents}
		\zh{\item 开发视频编解码新方法,参与标准制定}
		\en{\item Develop new video coding methods for standardisation purposes}
		\fr{\item Développement de nouvelles techniques du codage vidéo pour la normalisation}
	\end{rSubsection}
	
	\begin{rSubsection}{\zh{深圳大学}\en{Shenzhen University}\fr{Université de Shenzhen}}
		{2020-10 -- 2022-12}
		{\zh{生物医学工程学院,博士后研究员}\en{Post-doctoral Researcher at School of Biomedical Engineering}\fr{Chercheur postdoctoral à School of Biomedical Engineering}}
		{\zh{深圳}\lat{Shenzhen}}
		\zh{\item 基于视频脑电的认识神经科学研究}
		\en{\item Worked on electroencephalography (EEG) related cognitive neuroscience investigations}
		\fr{\item Recherche en neuroscience cognitive liées à l'électroencéphalographie (EEG)}
        \zh{\item 负责研发CT扫描中的颅内脑电电极定位算法}
		\en{\item Worked on project: localisation of intracranial EEG electrode contacts in CT scans}
		\fr{\item Localisation des contacts d'électrodes EEG intracrâniennes dans les scanners CT}
	\end{rSubsection}
	
	\begin{rSubsection}{\zh{香港大学}\en{The University of Hong Kong}\fr{Université de Hong Kong}}
		{2016-07 -- 2019-06}
		{\zh{电机电子工程系,研究助理}\en{Research Assistant at Department of Electrical and Electronic Engineering}\fr{Assistant de recherche à Departement of Electrical and Electronic Engineering}}
		{\zh{香港}\lat{Hong Kong}}
        \zh{\item 研发三维视频编解码方法}
		\en{\item Researched on 3D video coding}
		\fr{\item Codage de vidéos 3D}
        \zh{\item 自行开发的深度图压缩方法的效能超越同期的3D-HEVC}		
		\en{\item Depth map codec outperformed 3D-HEVC}
		\fr{\item Notre codec de carte de profondeur surpassait le 3D-HEVC à cette époque}
	\end{rSubsection}
	
	\begin{rSubsection}{\zh{万维数码}\lat{Marvel Digital Limited}}
		{2015-08 -- 2017-12}
		{\zh{工程咨询师}\lat{Consultant}}
		{\zh{香港}\lat{Hong Kong}}
        \zh{\item 参与三维视频编解码标准竞投,撰写专利文书}
		\en{\item Developed 3D video coding methods and deal with intellectual property paperworks} 
		\fr{\item Développement des méthodes de codage vidéo 3D et traité des documents de propriété intellectuelle}
	\end{rSubsection}
	
	\begin{rSubsection}{\zh{香港大学}\en{The University of Hong Kong}\fr{Université de Hong Kong}}
		{2011-08 -- 2015-08}
		{\zh{电机电子工程系,研究助理}\en{Research Assistant at Department of Electrical and Electronic Engineering}\fr{Assistant de recherche à Departement of Electrical and Electronic Engineering}}
		{\zh{香港}\lat{Hong Kong}}
        \zh{\item 参与中国数字音视频编解码技术标准项目(AVS)的研发}
		\en{\item Worked in collaboration with Audio Video coding Standard workgroup (AVS) of China}
		\fr{\item Travaillé en collaboration avec le groupe de travail sur l'Audio/Video coding Standards (AVS) de la Chine}
        \zh{\item 研发视频编解码加速,三维视频内容生成方法}
		\en{\item Focused on video codec design and acceleration, 3D video content generation}
		\fr{\item Axé sur la conception et l'accélération des codecs vidéo, la génération de contenu vidéo 3D}
        \zh{\item AVS-1在ARM+DSP平台上可以达到实时的编解码}
		\en{\item Achieved real-time AVS video encoding and decoding in ARM+DSP low power platform}
		\fr{\item Réalisé un encodage et un décodage vidéo AVS en temps réel sur une plateforme ARM+DSP basse consommation}
        \zh{\item 开发半自动的图像分割和抠图工具}		
		\en{\item Developed a semi-automatic image segmentation tool}
		\fr{\item Développé un outil de segmentation d'images semi-automatique}
	\end{rSubsection}
	
	\begin{rSubsection}{\zh{晶门科技}\lat{Solomon Systech Limited}}
		{2009-06 -- 2010-05}
		{\zh{设计工程部,工程实习生}\en{Engineering Trainee at Design Engineering Department}\fr{Stagiaire en ingénierie à Design Engineering Departement}}
		{\zh{香港}\lat{Hong Kong}}
        \zh{\item 移动端显示设备上的快速图像编解码}
		\en{\item Worked on image compression and coding algorithms for mobile device display system}
		\fr{\item Travaillé sur des algorithmes de compression et de codage d'images pour les systèmes d'affichage des appareils mobiles}
        \zh{\item 参与开发和测试新微控制器产品的汇编环境等工具,测试指令集}
		\en{\item Designed specifications and develop supporting softwares for new microcontroller product}
		\fr{\item Conçu des spécifications et développé des logiciels de support pour un nouveau produit de microcontrôleur}
	\end{rSubsection}
\end{rSection}

%----------------------------------------------------------------------------------------
%	PROJECTS SECTION
%----------------------------------------------------------------------------------------

\begin{rSection}{\zh{参与项目}\en{Projects}\fr{Projets}}
    \begin{rSubsection}{\zh{音视频编解码技术研究}\en{Audio and video coding research}\fr{Recherche sur le codage audio et video}}
		{2023-01 -- \zh{现在}\en{Now}\fr{Aujourd'hui}}{}{}
        \zh{\item 基于VVC的技术改进}
        \en{\item Technical development based on VVC standard}
        \fr{\item Développement technique basé sur le standard VVC}
        \zh{\item 参与JVET和AVS的提案活动和会议}
        \en{\item Participate in proposal activities and meetings of JVET and AVS}
		\fr{\item Participation aux activités de proposition et aux réunions de JVET et d'AVS}
    \end{rSubsection}
	
	\iffalse
	\begin{rSubsection}{\zh{颅内视频脑电智能计算}\en{Intracranial Video-electroencephalography}\fr{Vidéo-électroencéphalographie intracrânienne}}
		{2022-04 -- 2022-10}{}{}
        \zh{\item 主要负责的工作是从病人头颅CT扫描中寻找植入的脑电电极}
		\en{\item Localised EEG electrode contacts from CT scans for pre-surgical planning}
		\fr{\item Localisation des contacts d'électrodes EEG des images CT pour la planification pré-chirurgicale}
        \zh{\item 初步实现电极的检测、定位、自动分组}
		\en{\item Primarily achieved localisation and automatic grouping}
		\fr{\item Réalisation principalement de la localisation et le regroupement automatique}
	\end{rSubsection}
	\fi

	\begin{rSubsection}{\zh{辅助视频数据压缩和视角生成系统}\en{Auxiliary Video Data Compression and View Synthesis System}\fr{Compression des données vidéo auxiliaires et synthèse de vues}}
		{2015-05 -- 2018-12}{}{}
        \zh{\item 负责三维视频中的深度图压缩方法的设计,并开发了一套原创的深度图像和视频的压缩方法}
		\en{\item Designed compression algorithms for depth maps}
		\fr{\item Conception d'algorithmes de compression pour les images et les vidéos de profondeur}
        \zh{\item 在不依赖纹理图像的情况下压缩深度图像,其压缩效能与3D-HEVC相当或更佳}
		\en{\item Achieved a compression performance on par with 3D-HEVC or better without texture images}
		\fr{\item Présentation d'une performance de compression comparable ou supérieure à celle du 3D-HEVC sans recourir aux images de texture}
        \zh{\item 为压缩方法在GPU上设计了一套算法流程,达到1080p视频实时解码}
		\en{\item Developed a GPU-accelerated software for 1080p real-time decoding}
		\fr{\item Développement d'un logiciel accéléré par GPU pour le décodage en temps réel en 1080p}
	\end{rSubsection}
		
	\begin{rSubsection}{\zh{三维视频内容生成及处理系统}\en{3D Video Content Generation and Processing System}\fr{Génération et traitement de contenu vidéo 3D}}
		{2013-05 -- 2015-08}{}{}
        \zh{\item 开发半自动的图像分割和抠图算法}
		\en{\item Developed a semi-automatic image segmentation and data management tool}
		\fr{\item Développement d'algorithmes de segmentation d'images}
        \zh{\item 为制作三维视频的合作方提供了一个给图像进行物体切割,分层,组织数据的操作工具}
		\en{\item Designed related image segmentation algorithms}
		\fr{\item Développement d'un outil de segmentation d'images semi-automatique et de gestion des données pour notre partenaire de collaboration}
	\end{rSubsection}
		
	\begin{rSubsection}{\zh{AVS-1视频编码标准在移动平台的实现}\en{Real-time AVS-1 Video Codec on Mobile Platform}\fr{Codec vidéo AVS-1 en temps réel sur plateforme ARM+DSP}}
		{2011-08 -- 2013-05}{}{}
        \zh{\item 将AVS的视频编解码移植到与移动设备功耗相当的ARM+DSP平台}
		\en{\item Ported AVS video codec onto mobile-class ARM+DSP platform}
		\fr{\item Portage du codec vidéo AVS sur une plateforme ARM+DSP de classe mobile}
        \zh{\item 达到720p视频的实时编码和解码}
		\en{\item Achieved real-time 720p encoding and decoding}
		\fr{\item Atteinte d'un encodage et d'un décodage en 720p en temps réel avec gestion de la charge de calcul et traitement asynchrone}
	\end{rSubsection}

\end{rSection}

%----------------------------------------------------------------------------------------
%	PUBLICATION AND PATENT SECTION
%----------------------------------------------------------------------------------------

\iffalse
\begin{rSection}{\zh{发表的论文与专利}\en{Publications and Patents}\fr{Publications et brevets}}

Z. Lin, \textbf{H. Qin} and S. C. Chan, ``A New Probabilistic Representation of Color Image Pixels and Its Applications,'' in \textit{IEEE Transactions on Image Processing}, vol. 28, no. 4, pp. 2037-2050, April 2019.

``A depth discontinuity-based method for efficient intra coding for depth videos'', WO 2017/020808, February 09, 2017.

``Systems and Methods for multiple layer representation of depth map for intra coding'', Hong Kong Short-term Patent Application No. 19124682.6

\end{rSection}
\fi

% %----------------------------------------------------------------------------------------
% %	TECHNICAL STRENGTHS SECTION
% %----------------------------------------------------------------------------------------

\begin{rSection}{\zh{其它技能}\en{Other Skills}\fr{Autres techniques}}

\begin{tabular}{ @{} >{\bfseries}l @{\hspace{2ex}}l }
\zh{人类语言}\en{Natural Languages}\fr{Langues naturelles} & 
\zh{普通话(母语),粤语(母语),英语(熟练流利),法语(爱好,会一点)}
\en{Mandarin (native), Cantonese (native), English (fluent), French (elementary)}
\fr{Mandarin (maternelle), Cantonais (maternelle), }\\
\fr{&Anglais (courant), Français (élémentaire)\\}
\zh{计算机语言}\en{Computer Languages}\fr{Langages informatiques} & C/C++, Python, MatLab \\
\zh{工具}\en{Tools}\fr{Outils} & Visual Studio, VSCode, Eclipse, Anaconda platform
\end{tabular}

\end{rSection}

%----------------------------------------------------------------------------------------
