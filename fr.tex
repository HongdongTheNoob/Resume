% !TEX program = xelatex
%%%%%%%%%%%%%%%%%%%%%%%%%%%%%%%%%%%%%%%%%
% Medium Length Professional CV
% LaTeX Template
% Version 2.0 (8/5/13)
%
% This template has been downloaded from:
% http://www.LaTeXTemplates.com
%
% Original author:
% Trey Hunner (http://www.treyhunner.com/)
%
% Important note:
% This template requires the resume.cls file to be in the same directory as the
% .tex file. The resume.cls file provides the resume style used for structuring the
% document.
%
%%%%%%%%%%%%%%%%%%%%%%%%%%%%%%%%%%%%%%%%%

%----------------------------------------------------------------------------------------
%	PACKAGES AND OTHER DOCUMENT CONFIGURATIONS
%----------------------------------------------------------------------------------------

\documentclass{resume} % Use the custom resume.cls style

\usepackage[left=0.75in,top=0.6in,right=0.75in,bottom=0.6in]{geometry} % Document margins

\usepackage{fontspec, xunicode, xltxtra}
\usepackage{xeCJK}

\setmainfont{Calibri}
\setCJKmainfont{HarmonyOS Sans SC}

\usepackage[ocgcolorlinks]{hyperref}
\hypersetup{
	colorlinks   = true, %Colours links instead of ugly boxes
	urlcolor     = blue, %Colour for external hyperlinks
	linkcolor    = blue, %Colour of internal links
	citecolor   = red %Colour of citations
}

\name{QIN Hongdong} % Your name
\address{(+86)13603087140 \\ hongdongdonald@gmail.com} % Your phone number and email

\begin{document}

%----------------------------------------------------------------------------------------
%	SECTION PROFIL
%----------------------------------------------------------------------------------------

\begin{rSection}{Profil}

Docteur en génie électrique et électronique et ingénieur principal en normalisation technique. Axé sur le développement du codage vidéo, les enquêtes sur les brevets et les pratiques de normalisation.

\end{rSection}

%----------------------------------------------------------------------------------------
%	SECTION FORMATION
%----------------------------------------------------------------------------------------

\begin{rSection}{Formation}

{\bf Université de Hong Kong} \hfill {Sep 2013 -- Juin 2020} \\ 
Doctorat \\
Titre de la thèse : \href{http://hdl.handle.net/10722/318421}{\textit{Novel Techniques for Depth Map Compression}}

{\bf Université nationale de Singapour} \hfill {Août 2008 -- Déc 2008} \\ 
Étudiant en échange, Génie électrique et informatique

{\bf Université polytechnique de Hong Kong} \hfill {Sep 2006 -- Août 2011} \\ 
Licence en ingénierie électronique et informatique, honneur du 1er classe, avec stage industriel de 12 mois

\end{rSection}

%----------------------------------------------------------------------------------------
%	SECTION EXPÉRIENCE PROFESSIONNELLE
%----------------------------------------------------------------------------------------

\begin{rSection}{Expérience professionnelle}
	\begin{rSubsection}{TCL Industries}{Jan 2023 -- Aujourd'hui}{Ingénieur senior, normalisation technique chez Eagle Lab}{Shenzhen}
		\item Effectuer des pré-recherches techniques sur les technologies de codage audio/vidéo et analyser les brevets pertinents
		\item Développer de nouvelles techniques de compression vidéo pour les demandes de brevets et la normalisation
	\end{rSubsection}
	
	\begin{rSubsection}{Université de Shenzhen}{Oct 2020 -- Déc 2022}{Chercheur postdoctoral à l'école de génie biomédical}{Shenzhen}
		\item Travaillé sur des investigations de neuroscience cognitive liées à l'électroencéphalographie (EEG)
		\item Travaillé sur le projet : localisation des contacts d'électrodes EEG intracrâniennes dans les scanners CT
	\end{rSubsection}
	
	\begin{rSubsection}{Université de Hong Kong}{Juil 2016 -- Juin 2019}{Assistant de recherche au Département de génie électrique et électronique}{Hong Kong}
		\item Travaillé sur le codage vidéo 3D
		\item Notre codec de carte de profondeur surpassait le 3D-HEVC à cette époque
	\end{rSubsection}
	
	\begin{rSubsection}{Marvel Digital Limited}{Août 2015 -- Déc 2017}{Consultant}{Hong Kong}
		\item Développé des méthodes de codage vidéo 3D et traité des documents de propriété intellectuelle 
	\end{rSubsection}
	
	\begin{rSubsection}{Université de Hong Kong}{Août 2011 -- Août 2015}{Assistant de recherche au département de génie électrique et électronique}{Hong Kong}
		\item Travaillé en collaboration avec le groupe de travail sur les normes de codage audio-vidéo (AVS) de la Chine
		\item Axé sur la conception et l'accélération des codecs vidéo, la génération de contenu vidéo 3D
		\item Réalisé un encodage et un décodage vidéo AVS en temps réel sur une plateforme ARM+DSP basse consommation
		\item Développé un outil de segmentation d'images semi-automatique
	\end{rSubsection}
	
	\begin{rSubsection}{Solomon Systech Limited}{Juin 2009 -- Mai 2010}{Stagiaire en ingénierie au département de conception}{Hong Kong}
		\item Travaillé sur des algorithmes de compression et de codage d'images pour les systèmes d'affichage des appareils mobiles
		\item Conçu des spécifications et développé des logiciels de support pour un nouveau produit de microcontrôleur
	\end{rSubsection}
\end{rSection}

%----------------------------------------------------------------------------------------
%	SECTION PROJETS
%----------------------------------------------------------------------------------------

\begin{rSection}{Projets}

	\begin{rSubsection}{Vidéo-électroencéphalographie intracrânienne}{Avr 2022 -- Oct 2022}{}{}
		\item Localisation des contacts d'électrodes EEG à partir des scanners CT pour la planification pré-chirurgicale
		\item Réalisé principalement la localisation et le regroupement automatique
	\end{rSubsection}

	\begin{rSubsection}{Système de compression des données vidéo auxiliaires et de synthèse de vues}{Mai 2015 -- Nov 2018}{}{}
		\item Conception d'algorithmes de compression pour les images et les vidéos de profondeur
		\item Présentation d'une performance de compression comparable ou supérieure à celle du 3D-HEVC sans recourir aux images de texture
		\item Développement d'un logiciel accéléré par GPU pour le décodage en temps réel en 1080p
	\end{rSubsection}
		
	\begin{rSubsection}{Système de génération et de traitement de contenu vidéo 3D}{Mai 2013 -- Août 2015}{}{}
		\item Développement d'algorithmes de segmentation d'images
		\item Développement d'un outil de segmentation d'images semi-automatique et de gestion des données pour notre partenaire de collaboration
	\end{rSubsection}
		
	\begin{rSubsection}{Codec vidéo AVS-1 en temps réel sur plateforme ARM+DSP}{Août 2011 -- Mai 2013}{}{}
		\item Portage du codec vidéo AVS sur une plateforme ARM+DSP de classe mobile
		\item Atteinte d'un encodage et d'un décodage en 720p en temps réel avec gestion de la charge de calcul et traitement asynchrone
	\end{rSubsection}

\end{rSection}

%----------------------------------------------------------------------------------------
%	SECTION PUBLICATIONS ET BREVETS
%----------------------------------------------------------------------------------------

\iffalse
\begin{rSection}{Publications et brevets}

Z. Lin, \textbf{H. Qin} and S. C. Chan, ``A New Probabilistic Representation of Color Image Pixels and Its Applications,'' in

 \textit{IEEE Transactions on Image Processing}, vol. 28, no. 4, pp. 2037-2050, April 2019.

``A depth discontinuity-based method for efficient intra coding for depth videos'', WO 2017/020808, February 09, 2017.

``Systems and Methods for multiple layer representation of depth map for intra coding'', Hong Kong Short-term Patent Application No. 19124682.6

\end{rSection}
\fi

%----------------------------------------------------------------------------------------
%	SECTION COMPÉTENCES TECHNIQUES
%----------------------------------------------------------------------------------------

\begin{rSection}{Compétences techniques}

\begin{tabular}{ @{} >{\bfseries}l @{\hspace{2ex}} l }
Langues naturelles & Mandarin (langue maternelle), Cantonais (langue maternelle), \\
 & Anglais (courant) \\
Langages informatiques & C/C++, MatLab, Python \\
Outils & Visual Studio, VSCode, Eclipse, plateforme Anaconda
\end{tabular}

\end{rSection}

%----------------------------------------------------------------------------------------

\end{document}
